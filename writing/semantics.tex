\subsection*{Denotational semantics}

Evaluating $e$ with argument $\targ$ and state $\tmem$ results in the value $\denotefv{e}{\targ,\tmem}$ and the updated state $\denotefs{e}{\targ,\tmem}$.\\
$\denotevalue{\cdot} : \mathrm{Value} \times \mathrm{State} \to \mathrm{Value}$\quad\quad
$\denotestate{\cdot} : \mathrm{Value} \times \mathrm{State} \to \mathrm{State}$\quad\quad
$\denotef{e}{\targ,\tmem} \eqdef \left(\denotefv{e}{\targ,\tmem},\denotefs{e}{\targ,\tmem}\right)$

$\mathrm{State} \eqdef \mathrm{Memory} \times \mathrm{Log}$\quad\quad
$\denotesub{m}{\cdot} \triangleq \mathrm{fst} \circ \denotestate{\cdot}$\quad\quad
$\denotesub{l}{\cdot} \triangleq \mathrm{snd} \circ \denotestate{\cdot}$

$\operatorname{malloc}$ takes a state and number of bytes $n$, finds a contiguous region of free memory starting at address $p$, and returns $p$ and an updated state where addresses $[p,p+n)$ are marked as used.
\begin{align*}
\denotef{\uop{Num}{n}}{\targ,\tmem} &\eqdef
  \left(n,\tmem\right)\\
%
\denotef{\uop{Bool}{b}}{\targ,\tmem} &\eqdef
  \left(b,\tmem\right)\\
%
\denotef{\mathrm{Arg}}{\targ,\tmem} &\eqdef
  \left(\targ,\tmem\right)\\
%
\denotefv{\bop{Add}{e_1}{e_2}}{\targ,\tmem} &\eqdef
  \denotefv{e_1}{\targ,\tmem} +
  \denotefv{e_2}{\targ,\denotefs{e_1}{\targ,\tmem}}\\
%
\denotefs{\bop{Add}{e_1}{e_2}}{\targ,\tmem} &\eqdef
  \denotefs{e_2}{\targ,\denotefs{e_1}{\targ,\tmem}}\\
%
\denotef{\trop{If}{e_c}{e_t}{e_e}}{\targ,\tmem} &\eqdef
  \begin{cases}
    \denotef{e_t}{\targ,\denotefs{e_c}{\targ,\tmem}} &
      \denotefv{e_c}{\targ,\tmem} = \top\\
    \denotef{e_e}{\targ,\denotefs{e_c}{\targ,\tmem}} &
      \denotefv{e_c}{\targ,\tmem} = \bot
  \end{cases}\\
%
\denotef{\bop{Switch}{e_{pred}}{\Vec{e}\ }}{\targ,\tmem} &\eqdef
  \denotef{\Vec{e}\ [\denotefv{e_{pred}}{\targ,\tmem}]}{\targ,\denotefs{e_{pred}}{\targ,\tmem}}\\
\begin{aligned}
\denote{\bop{Let}{e_{in}}{e_{out}}}&\\
\denotef{\bop{Let}{e_{in}}{e_{out}}}{\targ,\tmem}&
\end{aligned}
\ &\begin{aligned}
  &\eqdef \denote{e_{out}} \circ \denote{e_{in}}
  &&\text{\color{gray}(point-free version)}\\
  &\eqdef \denotef{e_{out}}{\denotef{e_{in}}{\targ,\tmem}}
  &&\text{\color{gray}($\eta$-expanded for clarity)}\\
\end{aligned}\\
%
\denotefv{\uop{Print}{e}}{\targ,\tmem} &\eqdef []\\
\denotefs{\uop{Print}{e}}{\targ,\tmem} &\eqdef
  (\denotefsub{m}{e}{\targ,\tmem},\;
   \append{\denotefsub{l}{e}{\targ,\tmem}}{\denotefv{e}{\targ,\tmem}})\\
%
\denotefv{\uop{Read}{e}}{\targ,\tmem} &\eqdef
  \denotefsub{m}{e}{\targ,\tmem}[\denotefv{e}{\targ,\tmem}]\\
%
\denotestate{\uop{Read}{e}} &\eqdef \denotestate{e}\\
%
\denotef{\bop{Alloc}{e}{\tau}}{\targ,\tmem} &\eqdef \operatorname{malloc}(\denotefs{e}{\targ,\tmem},\; \denotefv{e}{\targ,\tmem} * \operatorname{sizeof}(\tau))\\
%
\denotefv{\bop{Write}{e_p}{e_d}}{\targ,\tmem} &\eqdef []\\
%
\denotefsub{m}{\bop{Write}{e_p}{e_d}}{\targ,\tmem} &\eqdef
\denotefsub{m}{e_p}{\targ,\denotefs{e_d}{\targ,\tmem}}[\denotefv{e_p}{\targ,\tmem} \mapsto \denotefv{e_d}{\targ,\tmem}]\\
%
\denotefsub{l}{\bop{Write}{e_p}{e_d}}{\targ,\tmem} &\eqdef
\denotefsub{l}{e_p}{\targ,\denotefs{e_d}{\targ,\tmem}}\\
%
\denotefv{\uop{Single}e}{\targ,\tmem} &\eqdef [\denotefv{e}{\targ,\tmem}]\\
\denotestate{\uop{Single}e} &\eqdef \denotestate{e} \\
%
\denotefv{\bop{Concat\ Sequential}{e_1}{e_2}}{\targ,\tmem} &\eqdef
\append{\denotefv{e_1}{\targ,\tmem}}{\denotefv{e_2}{\targ,\denotefs{e_1}{\targ,\tmem}}}\\
%
\denotefs{\bop{Concat\ Sequential}{e_1}{e_2}}{\targ,\tmem} &\eqdef
\denotefs{e_2}{\targ,\denotefs{e_1}{\targ,\tmem}}\\
%
\denotef{\bop{DoWhile}{e_{in}}{e_{po}}}{\targ,\tmem} &\eqdef
\mathrm{let}\ (v_{in},\tmem') = \denotef{e_{in}}{\targ,\tmem}\ \mathrm{in}\\
&\hspace{1.3em}\mathrm{let}\ ((v_{pred}, v_{out}), \tmem'') = \denotef{e_{po}}{v_{in},\tmem'}\ \mathrm{in}\\
&\hspace{1.3em}\begin{cases}
   (v_{out}, \tmem'')
   & v_{pred} = \bot\\
   \denotef{\bop{DoWhile}{\mathrm{Arg}\ }{e_{po}}}{v_{out},\tmem''}
   & v_{pred} = \top\\
\end{cases}\\
&\hspace{1.3em}\text{\color{gray}(this should be written as a fixpoint instead)}\\
\denotef{\bop{Context}{(\uop{InLet}{e_{in}})}{e}}{\targ,\tmem} &\eqdef
\begin{cases}
  \denotef{e}{\targ,\tmem} & \exists \targ', \tmem' . \; \denotefv{e_{in}}{\targ',\tmem'} = \targ
\end{cases}\\
%
\denotef{\bop{Context}{(\bop{AfterWrite}{e_p}{e_d})}{e}}{\targ,\tmem} &\eqdef
\begin{cases}
  \denotef{e}{\targ,\tmem} & \exists \targ', \tmem' . \; \denotefs{\bop{Write}{e_p}{e_d}}{\targ',\tmem'} = \tmem
\end{cases}\\
%
\denotef{\bop{Context}{(\bop{InLoop}{e_{in}}{e_{po}})}{e}}{\targ,\tmem} &\eqdef
\begin{cases}
  \denotef{e}{\targ,\tmem} &
    \exists  \tmem' . \; (\targ,\tmem') \in  \operatorname{loop\_reachable}(e_{in},e_{po})
\end{cases}
\end{align*}
\begin{lstlisting}[escapeinside={(*}{*)}]
def loop_reachable((*$e_{in}$*), (*$e_{po}$*)):
  res = {(*$\denotef{e_{in}}{\targ,\tmem} \mid \targ \in \mathrm{Value}, \tmem \in \mathrm{State}$*)}
  fixpoint:
    for (*$(\targ, \tmem) \in$*) res:
      (*$(v_{pred}, v_{out}), \tmem' = \denotef{e_{po}}{\targ,\tmem}$*)
      if (*$v_{pred}$*):
        res.insert((*$(v_{out}, \tmem')$*))
  return res
\end{lstlisting}


\newpage

\subsection*{Big-step operational semantics}

$\bigstep{\strip{e}{\targ}{\tmem}}{\spair{v}{\tmem'}}$ means: with argument $\targ$ and state $\tmem$, $e$ evaluates to $v$ and the resulting state is $\tmem'$.

A state is a pair $(M, L)$, containing memory and a print log.

%% \inferax{E-Num}{\easvsd{\uop{Num}{n}}{\targ}{\tmem}{n}{\tmem}}

%% \inferax{E-Bool}{\easvsd{\uop{Bool}{b}}{\targ}{\tmem}{b}{\tmem}}

\inferrule{E-Add}{
  \easvs{e_1}{\targ}{\tmem}{v_1}{\tmem'} \andalso
  \easvs{e_2}{\targ}{\tmem'}{v_2}{\tmem''}}
{\easvsd
  {\bop{Add}{e_1}{e_2}}{\targ}{\tmem}
  {v_1 + v_2}{\tmem''}}

\inferrule{E-IfTrue}{
  \easvs{e_c}{\targ}{\tmem}{\top}{\tmem'} \andalso
  \easvs{e_t}{\targ}{\tmem'}{v_t}{\tmem''}}
{\easvsd
  {\trop{If}{e_c}{e_t}{e_e}}{\targ}{\tmem}
  {v_t}{\tmem''}}

\inferrule{E-IfFalse}{
  \easvs{e_c}{\targ}{\tmem}{\bot}{\tmem'} \andalso
  \easvs{e_e}{\targ}{\tmem'}{v_e}{\tmem''}}
{\easvsd
  {\trop{If}{e_c}{e_t}{e_e}}{\targ}{\tmem}
  {v_e}{\tmem''}}

\inferrule{E-Switch}{
  \easvs{e_{pred}}{\targ}{\tmem}{i}{\sigma'} \andalso
  \easvs{e_i}{\targ}{\sigma'}{v}{\sigma''}}
{\easvsd
  {\bop{Switch}{e_{pred}}{(e_1, \dots, e_n)}}{\targ}{\tmem}
  {v}{\tmem''}}

\inferrule{E-Let}{
  \easvs{e_{in}}{\targ}{\tmem}{v_{in}}{\tmem'} \andalso
  \easvs{e_{out}}{v_{in}}{\tmem'}{v_{out}}{\tmem''}}
{\easvsd
  {\bop{Let}{e_{in}}{e_{out}}}{\targ}{\tmem}
  {v_{out}}{\tmem''}}

\inferrule{E-Print}{
  \easvs{e}{\targ}{\tmem}{v}{(M,L)}}
{\easvsd
  {\uop{Print}{e}}{\targ}{\tmem}
  {[]}{(M,\append{L}{v})}}

\inferrule{E-Read}{
  \easvs{e}{\targ}{\tmem}{v}{(M,L)}}
{\easvsd
  {\uop{Read}{e}}{\targ}{\tmem}
  {M[v]}{(M,L)}}

\inferrule{E-Alloc}{
  \easvs{e}{\targ}{\tmem}{n}{(M,L)} \andalso
  (M', p) = \operatorname{malloc}(M, n * \operatorname{sizeof}(\tau))}
{\easvsd
  {\bop{Alloc}{e}{\tau}}{\targ}{\tmem}
  {p}{(M',L)}}

\inferrule{E-Write}{
  \easvs{e_p}{\targ}{\tmem}{v_p}{\tmem'} \andalso
  \easvs{e_d}{\targ}{\tmem'}{v_d}{(M,L)}}
{\easvsd
  {\bop{Write}{e_p}{e_d}}{\targ}{\tmem}
  {[]}{(M[v_p \mapsto v_d], L)}}

\inferrule{E-Single}{
  \easvs{e}{\targ}{\tmem}{v}{\tmem'}}
{\easvsd
  {\uop{Single}{e}}{\targ}{\tmem}
  {[v]}{\tmem'}}

\inferrule{E-ConcatSeq}{
  \easvs{e_1}{\targ}{\tmem}{v_1}{\tmem'} \andalso
  \easvs{e_2}{\targ}{\tmem'}{v_2}{\tmem''}}
{\easvsd
  {\bop{Concat\ Sequential}{e_1}{e_2}}{\targ}{\tmem}
  {\append{v_1}{v_2}}{\tmem''}}

\inferrule{E-DoWhileFalse}{
  \easvs{e_{in}}{\targ}{\tmem}{\targ'}{\tmem'} \andalso
  \easvs{e_{po}}{\targ'}{\tmem'}{[\bot,v]}{\tmem''}}
{\easvsd
  {\bop{DoWhile}{e_{in}}{e_{out}}}{\targ}{\tmem}
  {v}{\tmem''}}

\inferrule{E-DoWhileTrue}{
  \easvs{e_{in}}{\targ}{\tmem}{\targ'}{\tmem'} \andalso
  \easvs{e_{po}}{\targ'}{\tmem'}{[\top,\targ'']}{\tmem''} \andalso
  \easvs{\bop{DoWhile}{e_{in}}{e_{po}}}{\targ''}{\tmem''}{v}{\tmem'''}}
{\easvsd
  {\bop{DoWhile}{e_{in}}{e_{po}}}{\targ}{\tmem}
  {v}{\tmem'''}}

\inferrule{E-ContextInLet}{
  \easvs{e}{\targ}{\tmem}{v}{\tmem'} \andalso
  \exists\ \targ_2, \tmem_2, \tmem_3 .\;
    \easvs{e_{in}}{\targ_2}{\tmem_2}{\targ}{\tmem_3}}
{\easvsd
  {\bop{Context}{(\uop{InLet}{e_{in}})}{e}}{\targ}{\tmem}
  {v}{\tmem'}}

\inferrule{E-ContextAfterWrite}{
  \easvs{e}{\targ}{\tmem}{v}{\tmem'} \andalso
  \exists\ \targ_2, v_2, \tmem_2 .\;
    \easvs{\bop{Write}{e_p}{e_d}}{\targ_2}{\tmem_2}{v_2}{\tmem}}
{\easvsd
  {\bop{Context}{(\bop{AfterWrite}{e_p}{e_d})}{e}}{\targ}{\tmem}
  {v}{\tmem'}}

\inferrule{E-ContextInLoop}{
  \easvs{e}{\targ}{\tmem}{v}{\tmem'} \andalso
  \exists\ \tmem_2 .\;
  (\targ, \tmem_2) \in \operatorname{loop\_reachable}(e_{in}, e_{out})}
{\easvsd
  {\bop{Context}{(\bop{InLoop}{e_{in}}{e_{po}})}{e}}{\targ}{\tmem}
  {v}{\tmem'}}


\inferrule{E-BeforeWrite}{
  M' = \operatorname{delete}(M,v_p) \andalso
  \easvs{e_p}{\targ}{(M',L)}{v_p}{\tmem} \andalso
  \easvs{e}{\targ}{(M',L)}{v}{\tmem}}
{\easvsd
  {\bop{Context}{(\bop{BeforeWrite}{e_{p}})}{e}}{\targ}{(M,L)}
  {v}{\tmem}}

\begin{comment}
〈     e, a, del(s, v_addr)〉⇓〈v, s'〉
〈e_addr, a, del(s, v_addr)〉⇓〈v_addr, s''〉
--------------------------------------------
〈(BeforeWrite e_addr e), a, s〉⇓〈v, s〉

The top condition is the "more intiutive" part of the rule. But why also delete
v_addr when stepping e_addr? This is motivating by the following example:
  The address
    (Snd (Print (Read addr)) addr)
  can never be the first argument to BeforeWrite, because it reads from the
  address written to!

----- Example of BeforeWrite rewrites ---

; As there are no writes in the second argument of outer-write,
; the assumption can go to both children. Otherwise it would only go
; to the second argument.
(Write
  (Snd (Write addr x) addr)
  y)
~~>
(Write
  (BeforeWrite addr
    (Snd (Write addr x) addr))
  (BeforeWrite addr
    y))

; As there are no writes in the second argument of (Snd (Write addr x)) addr,
; the first assumption can go to both children.
(BeforeWrite addr
  (Snd (Write addr x) addr))
~~>
(Snd
  (BeforeWrite addr (Write addr x))
  (BeforeWrite addr addr))

; We detected a shadowed write (as both write to the same address)!
(BeforeWrite addr (Write addr x))
~~>
(Empty)
\end{comment}

\newpage
\subsection*{Abstract grammar}
\begin{minipage}[t]{0.45\linewidth}
  \begin{grammar}
    <basetype> ::= IntT \alt BoolT

    <type> ::= <basetype> \alt PointerT <type> \alt TupleT <basetype>*

    <bool> ::= $\top$ \alt $\bot$

    <order> ::= Parallel \alt Sequential

    <function> ::= Function <type> <expr>

    <assumption> ::= InLet <expr>
    \alt InLoop <expr> <expr>
    %% \alt InFunc <function>
    %% \alt InSwitch <\N> <expr>
    %% \alt InIf <bool> <expr>
    \alt AfterWrite <expr> <expr>
  \end{grammar}
\end{minipage}\hfill%
\begin{minipage}[t]{0.5\linewidth}
  \begin{grammar}
    <expr> ::= Arg <type>
    \alt Int <\N>
    \alt Bool <bool>
    \alt Empty
    \alt Add <expr> <expr>
    \alt Sub <expr> <expr>
    \alt Mul <expr> <expr>
    \alt LessThan <expr> <expr>
    \alt And <expr> <expr>
    \alt Or <expr> <expr>
    \alt Write <expr> <expr>
    %% \alt PtrAdd <expr> <expr>
    \alt Not <expr>
    \alt Print <expr>
    \alt Read <expr>
    \alt Get <expr> <\N>
    \alt Alloc <expr> <type>
    \alt Call <function> <expr>
    \alt Single <expr>
    \alt Concat <order> <expr> <expr>
    \alt Switch <expr> <expr>*
    \alt If <expr> <expr> <expr>
    \alt Let <expr> <expr>
    \alt DoWhile <expr> <expr>
    \alt Context <assumption> <expr>
  \end{grammar}%
\end{minipage}%

